\documentclass{beamer}
\usepackage{xparse}
\usepackage{fontspec}
\usepackage{amsmath}
\usepackage{unicode-math}
\usepackage{listings}
\def\clap#1{\hbox to 0pt{\hss#1\hss}}
\def\mathllap{\mathpalette\mathllapinternal}
\def\mathrlap{\mathpalette\mathrlapinternal}
\def\mathclap{\mathpalette\mathclapinternal}
\def\mathllapinternal#1#2{%
\llap{$\mathsurround=0pt#1{#2}$}}
\def\mathrlapinternal#1#2{%
\rlap{$\mathsurround=0pt#1{#2}$}}
\def\mathclapinternal#1#2{%
\clap{$\mathsurround=0pt#1{#2}$}}


\lstset{escapeinside=@@,language={Haskell},basicstyle=\ttfamily}
\NewDocumentCommand\Eq{m}{%
  $\mathtt{\overset{\mathclap{\text{#1}}}{=}}$%
}

\begin{document}

\begin{frame}[fragile]{Zu zeigen}
  Für alle Listen \lstinline{xs :: [Int]} gilt:

\begin{lstlisting}
sum (foo xs) = 2 * sum xs - length xs
\end{lstlisting}

\end{frame}


\begin{frame}[fragile]{Induktionsanfang (IA)}
  Sei \lstinline{xs = []}, dann:

\begin{lstlisting}
@\phantom{$=$}@ sum (foo [])@\pause@
@\Eq{(2)}@ sum []@\pause@
@\Eq{(6)}@ 0@\pause@
@$=$@ 2 * 0 - 0@\pause@
@\Eq{(6)}@ 2 * sum [] - 0@\pause@
@\Eq{(2)}@ 2 * sum [] - length []
\end{lstlisting}
\end{frame}


\begin{frame}[fragile]{Induktionsbehauptung (IB)}
  Es gibt eine beliebige, aber feste Liste \lstinline{xs'} für die gilt:

\begin{lstlisting}
sum (foo xs') = 2 * sum xs' - length xs'
\end{lstlisting}
\end{frame}


\begin{frame}[fragile]{Induktionsschritt (IS)}

  Für alle \lstinline{x::Int} gilt: Sei \lstinline{xs = (x:xs')}:

\begin{lstlisting}
@\phantom{$=$}@ sum (foo (x:xs'))@\pause@
@\Eq{(3)}@ sum (x : x : (-1) : foo (xs'))@\pause@
@\Eq{(7)}@ x + sum (x : (-1) : foo (xs'))@\pause@
@\Eq{(7)}@ x + x + sum ((-1) : foo (xs'))@\pause@
@\Eq{(7)}@ x + x + (-1) + sum (foo (xs'))@\pause@
@\Eq{IV}@ x + x + (-1) + 2 * sum xs' - length xs'@\pause@
@$=$@ 2 * x + 2 * sum xs' - 1 - length xs'@\pause@
@$=$@ 2 * (x + sum xs') - (1 + length xs')@\pause@
@\Eq{(7)}@ 2 * sum (x:xs') - (1 + length xs')@\pause@
@\Eq{(11)}@ 2 * sum (x:xs') - length (x:xs') @$\qed$@
\end{lstlisting}
\end{frame}

\end{document}
