\documentclass[a4paper,10pt]{scrartcl}

%\usepackage[ngerman]{babel}
%\usepackage{unicode-math}
%\usepackage[T1]{fontenc}
%\usepackage{amsmath}
%\usepackage{amsfonts}
%\usepackage{stmaryrd}
\usepackage{xparse}
\usepackage{fontspec}
\usepackage{amsmath}
\usepackage{mathtools}
\usepackage{unicode-math}
%Hier stehen die Kopf und Fußzeilen

\usepackage{scrpage2} 
%\pagestyle{scrheadings} 

%\ihead[scrplain-innen] {Peter Klausing}
%\chead[scrplain-zentriert] {\rule[-1mm]{16cm}{0,21mm}}	
%\ohead[scrplain-außen] {\today~-~Sommersemester 2015}
%\cfoot[scrplain-zentriert] {\rule{16cm}{0,21mm} \\ \thepage}

%\usepackage[doublespacing]{setspace}


\NewDocumentCommand\alphaconversion{}{\Rightarrow_α}
\NewDocumentCommand\betareduction{}{\Rightarrow_β}
%\NewDocumentCommand\la {}{\la ngle}
%\NewDocumentCommand\rangle{}{\ranglengle}



%Seitenränder
\usepackage{geometry}
\geometry{a4paper, top=25mm, left=25mm, right=25mm, bottom=30mm}

\title{- Übersicht Lambda-Kalkül -}
\author{Peter Klausing}
\date{\today}

\begin{document}

$λ$ bezeichnet die hier Menge aller Lambdaterme

\section*{Applikation und Abstraktion}
Applikation von $λ$-Terme: $t = t_{1}t_{2}$ mit $t_{1},t_{2} \in λ$
Applikationen sind linksassoziativ: $t_1 t_2 t_3 = ((t_1 t_2) t_3)$\\
Abstraktion von $λ$-Terme: $t = (λx.t_1)$ mit $t_1 \in λ$  \\
Verschachtelte Abstraktionen können abgekürzt werden:\\
$(λx_1.(λx_2.(λx_3.x_1 x_2 x_3))) = (λx_1 x_2 x_3.x_1 x_2 x_3)$

\section*{gebundenne Variablen GV und freie Variablen FV}
Sei $t \in λ$:
\begin{itemize}
  \item $t = x$ mit $x \in X$ (Menge der Variablen) $\Rightarrow$ $GV(t)=\emptyset $ und $FV(t)=\{x\}$
  \item $t = t_1 t_2$ mit $t_1,t_2 \in λ$ $\Rightarrow$ $GV(t)=GV(t_1) \cup GV(t_2)$ und $FV(t)=FV(t_1) \cup FV(t_2)$
  \item $t = (λx.t_1)$ mit $t_1 \in λ$ $\Rightarrow$ $GV(t)=\{x\} \cup GV{t_1}$ und $FV(t)=FV(t_1)/ \{x\}$
\end{itemize}

\section*{Substitution}
$ t \left[ x / s \right] $ mit $t,s \in λ$ und $x \in X$: Substition aller frei Vorkommen von $x$ in $t$ durch $s$\\
Bsp.: $(λx.xy(λy.y))\left[ y : (λz.z)\right] \Rightarrow (λx.x(λz.z)(λy.y))$

\section*{$β$ - Reduktion}
$β$ -Reduktion ist die Auflösung einer Applikation von einem Term $t_2$ auf einer Abstraktion $(λx.t_1)$.\\
Voraussetzung: $GV(t_1) \cup FV(t_2) = \emptyset$.\\
Es werden alle freien Vorkommen von $x$ in $t_1$ durch $t_2$ subtituiert.\\ 
$\overbrace{\underbrace{(λx.t_1)}_{\text{Abstraktion}} t_2}^{\text{Applikation}} \betareduction t_1 \left[x / t_2\right]$ \\
Zu beachten ist die implizite links Assozitivität
\begin{itemize}
  \item nicht $β$-reduzierbar da implizite Klammerung: $(x (λy.y) z )= ((x (λy.y)) z)$
  \item innerhalb Abstraktion reduzierbar:
  $(λx.(λy.yx)z) \betareduction (λx.(yx)\left[y/z\right]) \Rightarrow (λx.zx) $
\end{itemize}

\section*{$α$ - Konversion}
Umbenennung einer gebundenen Variable in einem Term.\\
$(λx.t_1) \alphaconversion (λx'.t_1\left[x/x'\right])$\\
Vorraussetzung: $x'$ kommt nicht in $t_1$ vor

\section*{Normalform}
Normalform ist erreicht wenn keine weitere $β$-Reduktion mehr möglich ist.\\
Es ist möglich mehrere $β$-Reduktionen auf einen $λ$-Term anzuwenden. Alle Folgen von $β$-Reduktionen ($\Rightarrow^{*}$) führen zur gleichen Normalform. (Konfluenz von $λ$-Termen)

\newpage

\section*{Zahlen, Funktionen und Wahrheitswerte im Lambda-Kalkül}
\begin{itemize}
  \item (Church) Zahlen: 
  $\langle 0 \rangle = (λxy.y)$, $\langle 1\rangle = (λxy.xy)$, $\langle 2\rangle = (λxy.x(xy))$, $\ldots \langle n\rangle = (λxy.\underbrace{x(\cdots (x(x}_{n}y))\cdots))$
  \item Wahrheitswerte: $\langle true\rangle = (λxy.x)$, $\langle false\rangle = (λxy.y)$
  \item Nachfolgerfunktion: $\langle succ\rangle = (λz.(λxy.x(zxy)))$ Bsp: $\langle succ\rangle\langle n\rangle \Rightarrow^{*} \langle n +1 \rangle$
  \item Vorgängerfunktion: $ \langle pred \rangle = (λk.k(λpu.u(\langle succ\rangle(p\langle true\rangle))(p\langle true\rangle))(λu.u\langle 0\rangle\langle 0\rangle) \langle false\rangle)$
  \item Ist Null: $\langle iszero \rangle = (λk.k(\langle true\rangle \langle false\rangle )\langle true\rangle )$ Bsp: $\langle iszero\rangle \langle 0\rangle  = \langle true\rangle $
  \item if-then-else: $\langle ite\rangle  = (λbxy.bxy)$\\
  $\langle ite\rangle  b x y \Rightarrow \left\{\begin{matrix} x \textsf{ wenn } b \Rightarrow^{*} \langle true\rangle  \\ y \textsf{ sonst } \end{matrix}\right.$
  \item Addition: $\langle Add\rangle  =(λzxy.\langle ite\rangle (\langle iszero\rangle x)y(\langle succ\rangle  (z(\langle pred\rangle x)y)))$\\
  $\langle add\rangle  = \langle Y\rangle  \langle Add\rangle $
  \item Multiplikation: $\langle Mult\rangle  = (λzxy.\langle ite\rangle (\langle iszero\rangle x)\langle 0\rangle (\langle add\rangle y(z(\langle pred\rangle x)y)))$\\
  $\langle mult\rangle  = \langle Y\rangle \langle Mult\rangle $
  
  \item Fixpunktkombinator: $\langle Y\rangle = (λh.((λy.h(yy))(λy.h(yy)))) $
\end{itemize}
\end{document}